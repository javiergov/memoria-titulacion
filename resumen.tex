\newpage
%\thispagestyle{empty}
\begin{center}
 \Large \textbf{Diseño e Implementación de Aplicación Cliente en Dispositivos iOS para la Reproducción de Medios en un Servidor de Streaming Timeshift}\\

\normalsize Memoria para optar al título de Ingeniero Civil Electrónico, mención Computadores \\
\normalsize Javier González Ovalle \\
\normalsize Profesor Guía: Agustín González Valenzuela \\
\normalsize Agosto de 2013 %Junio de 2012

\Large \textbf{Resumen}

\end{center}
\normalsize
%\textbf{Objetivos del Trabajo:} \\
%\subsection*{Objetivos del Trabajo}
%\normalsize
%Siguiendo el esquema de AHidalgo.pdf

%Siguiendo el esquema de AHidalgo.pdf

%Primer parrafo, descripción breve de los objetivos
El objetivo de este trabajo buscó extender hacia los dispositivos móviles el servicio de \textit{streaming} por Internet desarrollado por la empresa AltaVoz S.A. La característica principal de este desarrollo permite al auditor cambiar el tiempo de la reproducción para ver programas pasados, repetir escenas de interés, o permitir pausar el contenido para ser visto más tarde, esta característica lleva el nombre comercial de \textit{timeshift}.\\


%Segundo párrafo, contextualizaciòn del problema
El servicio de \textit{stream timeshift} está disponible a través de la plataforma Web exclusivamente para computadores capaces de presentar contenido con Adobe Flash. El inconveniente de este sistema se debe al interés de posibles clientes de esta tecnología, por ejemplo canales de televisión, de llegar a la plataforma móvil, entiéndase teléfonos celulares del tipo \textit{smartphone} capaces de reproducir video. Debido a la incapacidad de la gran mayoría de los \textit{smartphones} de cargar el \textit{plug-in} Flash en sus navegadores Web se debe buscar una alternativa para presentar el contenido del \textit{stream} con la característica \textit{timeshift}.
\\

 
%Tercer y cuarto párrafo, que y como se realizó

En primera instancia se investigó el protocolo que utiliza el \textit{streaming timeshift} desarrollado por AltaVoz S.A. y se comparó con las opciones de reproducción que dispone el sistema operativo iOS para \textit{smartphones}, en base a ésto se propuso modificaciones al servidor del \textit{stream} de forma que el contenido fuera compatible con los dispositivos iOS. En segunda instancia se desarrolló una aplicación iOS con tal que los dispositivos iPhone e iPad reproduzcan los contenidos del servidor modificado además de permitier el control del tiempo de reproducción. Luego fue necesario implementar un sistema para compartir las marcas temporales integrando la red social Twitter. Finalmente fue necesario comprobar el comportamiento de lo desarrollado en distintos escenarios de conectividad a Internet con el dispositivo móvil.\\


%Palabras clave.
\textbf{Palabras Claves:} streaming, smartphone, apple, iOS, android, Twitter.
%sensores, ruteo, protocolos, algoritmos, energía.

