\newpage
\thispagestyle{empty}
\begin{center}
 \Large \textbf{Diseño e implementación de aplicación cliente en dispositivos iOS para la reproducción de medios en un servidor de streaming timeshift}\\

\normalsize Memoria para optar al título de Ingeniero Civil Electrónico, mención Computadores \\
\normalsize Javier González Ovalle \\
\normalsize Profesor Guía: Agustín González Valenzuela \\
\normalsize Junio de 2012

\Large \textbf{Resumen}

\end{center}
\normalsize
%\textbf{Objetivos del Trabajo:} \\
\subsection*{Objetivos del Trabajo}
\normalsize
%Siguiendo el esquema de AHidalgo.pdf

%Primer parrafo, descripción breve de los objetivos
Nuevos avances se han realizado en el manejo de transmisiones en vivo de medios como audio y/o video, que permiten emular la característica principal de dispositivos de sobremesa del tipo \textquotedblleft Programmable (Digital) Video Recorders\textquotedblright \ (PVR ó DVR), los cuales permite pausar, retroceder o avanzar, cuando es posible, la linea de tiempo de reproducción. Esto se logra gracias a aplicaciones que interactuan con servidores de flujo de datos de audio y video para lograr un registro de la reproducción, entregando al cliente comunicación eficaz para recibir datos y retornar comandos.\\ 

%Segundo párrafo, contextualizaciòn del problema
Se busca suplir la necesidad de un reproductor de audio y/o video que permita la interacción con un servidor de \textit{streaming} (flujo) de datos, con habilidad de \textit{Time-Shift} (cambio temporal). Esto es la habilidad de retroceder o avanzar (si es posible) en la linea de tiempo de la reproducción del audio o del video al que el usuario atiende.\\

Esto ya se logra en computadores de escritorio gracias a un reproductor de streaming basado en la tecnolog\'ia Flash de Adobe. Los reproductores embebidos en p\'aginas web reproducen el stream en base al protocolo Real Time Messaging Protocol (RTMP), el cual permite llamadas de control entre cliente y servidor para manejar la linea de tiempo. Sin embargo se busca llegar a un mayor público portando la capacidad de reproducci\'on en dispositivos que utilizan el sistema operativo iOS de Apple.\\

El objetivo de este trabajo se debe a que Apple por políticas internas, no apoya la distribuci\'on de Adobe Flash. Por lo tanto se busca implementar una soluci\'on que permita la reproducci\'on de la transmisi\'on con \textit{Time-Shift} y que adem\'as concuerde con las pol\'iticas de distribuci\'on para las aplicaciones iOS en la AppStore.
 
\large
%\textbf{Trabajo a Desarrollar:}\\
\subsection*{Trabajo a Desarrollar}
\normalsize
%Tercer y cuarto párrafo, que y como se realizó
El trabajo toma como punto de partida la investigaci\'on de los protocolos propuestos por Apple para la distribuci\'on de medios (audio y/o video). En base a estos protocolos buscar similitudes con el sistema de streaming para computadores de escritorio. Adem\'as se busca implementar un m\'etodo de marcaci\'on de tiempo en el stream de datos destinado a dispositivos m\'oviles, de forma que el servidor los env\'ie para que el cliente los interprete. Este cliente debe consistir en una aplicación que reciba y env\'ie comandos compatibles con el servidor Wowza para controlar la informaci\'on del medio a reproducir. Finalmente se busca diseñar una interfaz gr\'afica que represente la disponibilidad de tiempo y permita controlar la reproducción.\\

%\textbf{Evaluaciones a Realizar:}
\subsubsection*{Evaluaciones a Realizar}
\normalsize
\begin{itemize}
\item	Comprobar la transmisión y recepción de datos en HTTP Live Stream u otro protocolo compatible con iOS, a través de un servidor de streaming dedicado y un dispositivo con iOS.
\item	Comprobar el intercambio de mensajes de control a través de un enlace entre el cliente móvil y el servidor de streaming.
\item	Pruebas comparativas de la reproducción en dispositivo móviles y computadores de escritorio.
\end{itemize}

%\textbf{Resultados esperados:}\\
\subsubsection*{Resultados esperados}
\normalsize
Se pretende que la reproducción en dispositivos móviles de las transmisiones en vivo se asemejen a la experiencia de reproducción en computadores de escritorio.\\

%Palabras clave.
%	\textbf{Palabras Claves:} streaming, iOS, video, audio, apple, iPhone.
