\newpage
\thispagestyle{empty}
\begin{center}
 \Large \textbf{Evaluación, selección e implementación de un protocolo de ruteo para un sistema de recolección de datos basado en una red de sensores inalámbricos}\\

\normalsize Memoria para optar al título de Ingeniero Civil Electrónico, mención Computadores \\
\normalsize Luis Enrique Espinoza Severino \\
\normalsize Profesor Guía: Agustín González Valenzuela \\
\normalsize Noviembre 2010

\Large \textbf{Resumen}

\end{center}
\normalsize

%Siguiendo el esquema de AHidalgo.pdf

%Primer parrafo, descripción breve de los objetivos
El objetivo de este trabajo es implementar un protocolo de ruteo multi-hop que permita a un sistema de recolección de datos, construido sobre una red de sensores inalámbricos, transportar los datos generados por los sensores a un punto de recolección en la red.\\ 

%Segundo párrafo, contextualizaciòn del problema
El sistema de recolección objetivo está orientado principalmente al monitoreo de variables físicas relacionadas con la agricultura como la humedad y la temperatura de los suelos, en zonas de difícil acceso y que no cuentan con redes de distribución eléctrica. La red de sensores inalámbricos que se utiliza para este trabajo está implementada sobre la plataforma de hardware TmoteSky y la plataforma de software TinyOS. En este contexto, se investigan e implementan protocolos que permitan extender el tiempo de autonomía de los nodos al orden de meses.\\
 
%Tercer y cuarto párrafo, que y como se realizó
Para comenzar, se realiza una investigación de los algoritmos de ruteo para redes de sensores que se han propuesto en la literatura y los protocolos que se han implementado en algunas plataformas. Posteriormente, se realiza una implementación del protocolo PEGASIS y un diseño e implementación de un protocolo de ruteo con topología de árbol que utiliza el componente ''Low Power Listening'' para reducir el consumo energético. Finalmente, se realiza una evaluación energética en la que se determina que con la utilización del componente ''Low Power Listening'' es posible implementar una red con un tiempo de autonomía energética del orden requerido.\\

%Palabras clave.
\textbf{Palabras Claves:} redes, sensores, ruteo, protocolos, algoritmos, energía.
