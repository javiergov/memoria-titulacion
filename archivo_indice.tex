%archivo_indice.tex
\chapter{Diseño e Implementación}
El siguiente capitulo consta de las bases necesarias para realizar la solución.
%\part{Primera parte}
\section{Protocolo HTTP Live Stream}
	\subsection{Especificación}
		\subsubsection{Draft Apple}
		\subsubsection{Caracteristicas a utilizar}
			
	\subsection{Herramientas dispuestas por Apple}
		\subsubsection{id3Tag Generator}
		\subsubsection{MediaStream Segmenter}
		\subsubsection{MediaFile Segmenter}
	\subsection{Ejemplo de transmisión}
\clearpage	
\section{Reproducción y Control de Stream de Video}
	\subsection{Servidor HTTP}
		\subsubsection{Obtención del Stream}	
		\subsubsection{Sementación del Video}
		\subsubsection{Playlist Dispatcher}
	\subsection{Cliente iOS}
		\subsubsection{Recopilación de datos del Stream}
		\subsubsection{Reproducción con AV Framework}
	\subsection{Intercambio de información entre componentes}
\clearpage
\section{Interfaz Gráfica}
	\subsection{Apple Design Guidelines}
	\subsection{Componentes de UIKit}
		\subsubsection{UIDatePicker}
		\subsubsection{UIButton}
		\subsubsection{...etc}
\clearpage
\section{Integración Redes Sociales}
	\subsection{Twitter}
		\subsubsection{API Twitter}
		\subsubsection{Sharekit vs. Twitter Framework}
		\subsubsection{Post en Twitter}
		\subsubsection{Seguimiento de Hashtag}
		\subsubsection{Extracción de información de un Tweet}
	\subsection{Bit.ly}
		\subsubsection{API Bit.ly}
		\subsubsection{Seguimiento de un Hipervínculo}
		\subsubsection{Expansión de atajo bit.ly}
\clearpage
\section{Registro en iOS}
	\subsection{Schemes}
	\subsection{Redireccionamiento via Web}
		\subsubsection{PHP Script}
		\subsubsection{Respuestas según User Agent}
		
\chapter{Pruebas de funcionamiento}
% datos en el camino, seguimiento de un tweet, pantallazos de debug
\section{Cambio de tiempo y/o canal}
\section{Proceso corriendo en fondo} % background
\section{Cambio de ancho de banda}
	\subsection{WiFi}
	\subsection{3G}
\section{Cambio de transmisión mediante Twitter}
	\subsection{Dentro de la aplicación}
	\subsection{Scheme registrado en iOS}
\section{Enlaces Twitter en otros dispositivos}
	\subsection{PC Escritorio}
		\subsubsection{Chrome}
		\subsubsection{Safari}
		\subsubsection{Firefox}
		\subsubsection{Opera}
		\subsubsection{Internet Explorer}
	\subsection{Android y otros móviles incompatibles}
\section{Comparación con SocialStream Flash}

\chapter{Conclusiones y trbajo futuro}
% pagina y media no mas de 2
	\section{futuro}
	%EXTENSIONES PARA hacerle, que no las haré pero mejorarian el producto en si, facebook, otras redes, etc
	% todo list
	\section{conclusiones}
	% lo mas importante del trabajo 
	% las decisiones claves y el por que de estas
	% decisiones malas tambien
	
	
%60 paginas MAX HASTA CONCLUSIONES
\chapter{Anexos}
	\section{Hardware}
		\subsection{Macintosh}
		\subsection{iPhone, iPad, iPod Touch}
	\section{Sistemas operativos}
		\subsection{OSX}
		\subsection{iOS}
	\section{Frameworks}
	%entorno de desarrollo utilizado
	
\chapter{Bibliografia}
	% refencias de lo estudiado y utilizado para el trabajo