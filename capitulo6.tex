\chapter{Trabajo futuro y conclusiones}

Luego de completar el trabajo, se tiene una clara visión de las mejoras para una próxima iteración de este trabajo y las conclusiones que se obtienen según los objetivos iniciales planteados.
\section{Trabajo Futuro}

En el protocolo de árbol de recolección implementado en capítulo 5, es posible que se generen círculos en las asociaciones. Estos círculos causarían que los mensajes generados por un nodo en vez de llegar al Gateway, transiten por este círculo y nunca salgan de él. Para una futura revisión del protocolo, sería conveniente implementar un mecanismo para la detección de estos círculos, para dar mayor robustez a la implementación.\\

Si bien la implementación del Protocolo PEGASIS no logró cumplir con los requerimientos de este trabajo, es posible, que aplicar un esquema similar sobre el protocolo de árbol de recolección que se implementó en el capítulo 5, permita un mejor desempeño energético de la red. Esto se debe a que las ideas de utilizar la redundancia (en términos de que más de un nodo puede rutear los mensajes de un nodo hijo, no sólo el padre) para repartir el tráfico en los nodos y la variación dinámica en la potencia de transmisión, podrían optimizar el consumo de corriente de los dispositivos y repartirlo por la red.\\

La aplicación WSNSniffer, que se desarrolló para verificar el comportamiento de la red, fue de gran utilidad. Para una próxima revisión, este programa puede ser llevado a otro nivel, incorporando la capacidad de detectar los mensajes de protocolos como CTP, Dissemination, FTSP y las aplicaciones que se distribuyen como ejemplo en el sistema TinyOS. También, se podría agregar una funcionalidad que permita agregar dinámicamente los mensajes que se desea detectar. Esta herramienta podría ser de gran ayuda para el estudio de las redes de sensores inalámbricos y para futuros desarrollos.

%¿Qué escribio alviña en sus conclusiones?

%Primer párrafo: en el primer parrafo concluye las debilidades de los actuales sistemas de monitoreo.

%Segundo párrafo: Habla sobre el porque de su solucion y que conocimientos aplicó para llevarla a cabo.

%Tercer párrafo: Explica que que soluciona su sistema y qué es lo que hace.

%Cuarto Habla de demora y dificultades en el desarrollo.

%Quinto y último párrafo: Habla de las proyecciones a futuro de su trabajo.
\section{Conclusiones}
%Concluciones acerca de la investigación.
Los algoritmos como LEACH, APTEEN y PEGASIS, no pueden por si solos reducir el consumo energético total de un dispositivo como el TMoteSky, para alcanzar un tiempo de operación de nodos del orden de meses. Para lograr esto, es necesario utilizar otras técnicas como Low Power Listening.\\

Los protocolos de ruteo implementados para redes de sensores inalámbricos que utilizan el TmoteSky y TinyOS (CTP y Dissemination), no están diseñados para cuidar el recurso energético del dispositivo. Actualmente, no están diseñados para funcionar en conjunto con el componente Low Power Listening que es el que permite reducir el consumo de corriente por el problema de ''Idle Listening'', y así, extender el tiempo de operación de los dispositivos al orden de meses. \\

Como se verificó en las secciones 5.2 y 5.3, se implementó satisfactoriamente un protocolo que rutea los mensajes hacia el Gateway a través de un árbol de recolección. El protocolo tolera la desaparición de nodos y utiliza el componente Low Power Listening. Este último permite: desplegar una red de sensores, autónomos energéticamente, en un campo sin distribución eléctrica; los dispositivos permitirían recolectar mediciones, por ejemplo, de humedad y temperatura, realizando esta operación por al menos un mes y hasta 3 meses (de acuerdo con los resultados obtenidos en los escenarios utilizados). La solución propuesta permitiría desarrollar una solución concreta al problema de recolección de datos basado en una red de sensores inalámbricos, utilizando un esquema multi hop, sobre una red asincrónica y permitiendo tiempo de operación de los nodos del orden de 1 a 3 meses, según el hardware energético que se utilice.