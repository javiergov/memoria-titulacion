
\chapter{Conclusiones y trabajo futuro}
% pagina y media no mas de 2
explicar que apenas se terminó, el proyecto se usó como base para janus cooperativa, pero solo radio.
es posible generar productos en base al reproductor, se puede mejorar la interfaz gráfica con preparativos en base a diseño de interfaces junto a especialistas como diseñadores gráficos.
además el trabajo sienta base para la compartición de contenido en vivo a través de plataformas móviles, muy similar a la tendencia actual de las llamadas smart tv que tambien incorporan redes sociales.



	\section{futuro}
	mostrar timeline nueva, desarrollo en la empresa.
	preview en popup
	redes sociales facebook
	destacados del editor
	inclusión de favoritos
	boton rtune
	
	
	%EXTENSIONES PARA hacerle, que no las haré pero mejorarian el producto en si, facebook, otras redes, etc
	% todo list
	\section{conclusiones}
	el trabajo precisó de investigación a fondo de las tecnologias de Apple Inc. adentrarse en el sistema iOS y su riguroso cuidado para las apps, estudiar las tecnologias en desarrollo por parte de ellos, leer los drafts en desarrollo de http live streaming, lo necesario para especificar el dispatcher que es algo fuera de lo normal para el protocolo.
	el estudio del SDK, las clases de stanford, guidelines.

% claves	
	usar http live streaming.
	por qué tuve que insistir en apegarse a las recomendaciones de apple siendo que se preferian otras en altavoz.
	compatibilidad desde ios 5.0 en adelante	solo por pto flotante en lista de segmentos.
% malas
	mala decision fue investigar portar rtmp, para darse cuenta que no sería permitido por apple, utilizar segmentos de 5 segundos, ventanas de tranmisión muy angostas y también muy amplias. el no 
	
		
	% lo mas importante del trabajo 
	% las decisiones claves y el por que de estas
	% decisiones malas tambien
	
	
%60 paginas MAX HASTA CONCLUSIONES