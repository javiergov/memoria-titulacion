\newpage
\thispagestyle{empty}
\begin{center}
 \Large \textbf{Evaluation, selection and implementation of a routing protocol for a data collection system based on a wireless sensor network}\\

%\normalsize Memoria (buscar traducción de memoria) para optar al título de Ingeniero Civil Electrónico, mención Computadores \\
\normalsize Luis Enrique Espinoza Severino \\
%\normalsize Profesor : Agustín González Valenzuela \\
\normalsize November 2010

\Large \textbf{Abstract}

\end{center}
\normalsize
The goal of this work is to implement a multi-hop routing protocol that allow to a data collection system, built on a wireless sensor network, move the data generated by sensors to a collection point in the network.\\

The target collection system is oriented to monitoring of physical variables related with agriculture, such as soil humidity and temperature, in areas with difficult access and without electricity distribution networks. The wireless sensor network used in this work is implemented on the TmoteSky hardware and TinyOS software. In this context, energy-aware protocols were investigated and implemented.\\

To begin, in this work is conducted an investigation in routing algorithms proposed in the literature and protocols that have been implemented. Subsequently, is performed an implementation of the PEGASIS protocol. Furthermore, is performed a design and implementation of a routing protocol with a tree topology using the Low Power Listening component to reduce energy consumption. Finally, is conducted an energy consumption evaluation in which is determined that use of the Low Power Listening component enable to deploy a network with the lifetime of the order required.


