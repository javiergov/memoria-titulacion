\newpage
%\thispagestyle{empty}
\begin{center}
% Diseño e implementación de aplicación cliente en dispositivos iOS para la reproducción de medios en un servidor de streaming timeshift
 \Large \textbf{Design and Implementation of a Client iOS Application for a Streaming Media Server with Timeshift Capabilities}\\
  

%\normalsize Memoria (buscar traducción de memoria) para optar al título de Ingeniero Civil Electrónico, mención Computadores \\
\normalsize Javier Cristóbal González Ovalle \\
%\normalsize Profesor : Agustín González Valenzuela \\
\normalsize August 2013

\Large \textbf{Abstract}

\end{center}
\normalsize

%Primer parrafo, descripción breve de los objetivos
The goal of this project is to extend for mobile devices the streaming media services of the company AltaVoz S.A. The main feature of their services is to give the audience a certain control over what they are watching, by being able to seek to another time of playback. This capability gives to the user the freedom of watching past shows, repeat highlights, or pause the moment for later playback. The feature is being promoted as \textit{timeshift}.\\



%Segundo párrafo, contextualizaciòn del problema
The \textit{timeshift} streaming media service is currently available for the Web platform, exclusively for computers able to display Adobe Flash content. The main issue for this service comes from the interest of potential clients which are focused on broadcasting their content to mobile devices, specifically smartphones capable of media playback. Since a great amount of devices are not able to load the Adobe Flash plug-in in their Web Browsers, there is a necessity for an alternative method of media playback with the \textit{timeshift} feature.\\

 
%Tercer y cuarto párrafo, que y como se realizó


Research started with the current protocols being used at the moment for the service developed by AltaVoz S.A., these were compared with the options for streaming media playback given by Apple Inc. for their iOS mobile operating system, the results meant several points which were needed to be modified in the server side, this way the server would deliver compatible streaming media to iOS devices. Secondly was the development of an iOS application able to play the content delivered by the modified \textit{timeshift} server, also featuring a simple user interface for the \textit{timeshift} capability.
Later was needed the implementation of a sharing system for time stamps, this was succeed by integrating the social network Twitter. To conclude it was necessary to check the behavior for different scenarios of Internet connectivity in the mobile device. 
\\

