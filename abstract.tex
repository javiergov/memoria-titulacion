\newpage
\thispagestyle{empty}
\begin{center}
% Diseño e implementación de aplicación cliente en dispositivos iOS para la reproducción de medios en un servidor de streaming timeshift
 \Large \textbf{Design and implementation of a client iOS application for playback from a streaming media server with time-shift capabilities}\\
  
 %\Large \textbf{Evaluation, selection and implementation of a routing protocol for a data collection system based on a wireless sensor network}\\

%\normalsize Memoria (buscar traducción de memoria) para optar al título de Ingeniero Civil Electrónico, mención Computadores \\
\normalsize Javier Cristóbal González Ovalle \\
%\normalsize Profesor : Agustín González Valenzuela \\
\normalsize June 2013

\Large \textbf{Abstract}

\end{center}
\normalsize

%Primer parrafo, descripción breve de los objetivos
The goal of this assignment is to extend for mobile devices the streaming media services of the company AltaVoz S.A. The main feature of their services is to give the audience a certain control over what they're watching, by being able to seek to another time of playback. This capability gives to the user the freedom of watching past shows, repeat highlights, or pause the moment for later playback. The feature is being promoted as \textit{timeshift}.\\

%El objetivo de este trabajo buscó extender hacia los dispositivos móviles el servicio de \textit{streaming} por Internet desarrollado por la empresa AltaVoz S.A. La característica principal de este desarrollo permite al auditor cambiar el tiempo de la reproducción para ver programas pasados, repetir escenas de interés, o permitir pausar el contenido para ser visto más tarde, esta característica lleva el nombre comercial de \textit{timeshift}.\\


%Segundo párrafo, contextualizaciòn del problema
The \textit{timeshift} streaming media service is currently available for the web platform, exclusively for computers able to display Adobe Flash content. The main issue for this service comes from the interest of potential clients which are focused on broadcasting their content to mobile devices, specifically smartphones capable of media playback. Since a great amount of devices aren't able to load the Adobe Flash plug-in in their Web Browsers, there is a necessity for an alternative method of media playblack with the \textit{timeshift} feature.\\

%El servicio de \textit{stream timeshift} está disponible a través de la plataforma Web exclusivamente para computadores capaces de presentar contenido con Adobe Flash. El inconveniente de este sistema se debe al interés de posibles clientes de esta tecnología, por ejemplo canales de televisión, de llegar a la plataforma móvil, entiéndase teléfonos celulares del tipo \textit{smartphone} capaces de reproducir video. Debido a la incapacidad de la gran mayoría de los \textit{smartphones} de cargar el \textit{plug-in} Flash en sus navegadores Web se debe buscar una alternativa para presentar el contenido del \textit{stream} con la característica \textit{timeshift}. \\
 
%Tercer y cuarto párrafo, que y como se realizó

%En primera instancia se investigó el protocolo que utiliza el \textit{streaming timeshift} desarrollado por AltaVoz S.A. y se comparó con las opciones de reproducción que dispone el sistema operativo iOS para \textit{smartphones}, en base a ésto se propuso modificaciones al servidor del \textit{stream} de forma que el contenido fuera compatible con los dispositivos iOS. En segunda instancia se desarrolló una aplicación iOS con tal que los dispositivos iPhone e iPad reproduzcan los contenidos del servidor modificado además de permitier el control del tiempo de reproducción. Luego fue necesario implementar un sistema para compartir las marcas temporales integrando la red social Twitter. Finalmente fue necesario comprobar el comportamiento de lo desarrollado en distintos escenarios de conectividad a Internet con el dispositivo móvil.\\

Research started with the current protocols being used at the moment for the service developed by AltaVoz S.A., these were compared with the options for streaming media playback given by Apple Inc. for their iOS mobile operating system, the results meant several points which were needed to be modified in the server side, this way the server would deliver compatible streaming media to iOS devices. Second was the development of an iOS application able to play the content delivered by the modified \textit{timeshift} server, also featuring a simple user interface for the \textit{timeshift} capability.
Later was needed the implementation of a sharing system for timestamps, this was succeed by integrating the social network Twitter. Finally it was necessary to check the behavior for different scenarios of Internet connectivity in the mobile device. 
\\


% ------------------------------------------------

%The goal of this work is to implement a multi-hop routing protocol that allow to a data collection system, built on a wireless sensor network, move the data generated by sensors to a collection point in the network.\\

%The target collection system is oriented to monitoring of physical variables related with agriculture, such as soil humidity and temperature, in areas with difficult access and without electricity distribution networks. The wireless sensor network used in this work is implemented on the TmoteSky hardware and TinyOS software. In this context, energy-aware protocols were investigated and implemented.\\

%To begin, in this work is conducted an investigation in routing algorithms proposed in the literature and protocols that have been implemented. Subsequently, is performed an implementation of the PEGASIS protocol. Furthermore, is performed a design and implementation of a routing protocol with a tree topology using the Low Power Listening component to reduce energy consumption. Finally, is conducted an energy consumption evaluation in which is determined that use of the Low Power Listening component enable to deploy a network with the lifetime of the order required.


