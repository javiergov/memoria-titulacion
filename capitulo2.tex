\chapter{Investigación sobre protocolos de ruteo}
%Breve introducción


\section{Algoritmos de ruteo propuestos en la literatura}
%Introducción

%Descripción de algunos algoritmos
%Protocolo LEACH
\subsection{Low Energy Adaptive Clustering Hierarchy}

%En general, ¿Cómo funciona?

%Descripción detallada

%Descripción del modelo energético
\subsubsection{Descripción del modelo energético utilizado para el diseño de LEACH}


%Descripción del funcionamiento
\subsubsection{Descripción del funcionamiento}

%Protocolo PEGASIS
\subsection{Power-Efficient Gathering in Sensor Information Systems}
%¿Qué es PEGASIS?

%¿Cómo funciona?
% ...

\subsubsection{Descripción del funcionamiento}
\subsection{Threshold sensitive Energy Efficient sensor Network protocol}

\subsubsection{Descripción del funcionamiento}

%Protocolo HEED
\subsection{Hybrid, Energy-Efficient, Distributed}


\subsection{Descripción del funcionamiento}

\section{Protocolos de ruteo implementados en TinyOS}

\subsection{Dissemination }
%¿Qué es diseminación?

%¿Cómo funciona?

%Carácterísticas
%Me aburrió también


\subsection{TYMO}


%Se puede rellenar con un poquito más cierto....

\subsection{Collection Tree Protocol}
%¿Qué es CTP?

%¿Cómo funciona?

\subsection{Berkeley Low-Power IP Stack}
%Damn it, este protocolo es bakan y cumple todo.
%¿Qué es BLIP?

%¿Cómo funciona?


%Agregar características ---ya me aburrí.

\section{Selección de un esquema de recolección}
%De lo general a lo particular

%Sección para comparar protocolos propuestos e implementados sobre sus métricas
\subsection{Análisis comparativo de resultados}

%También, en términos de energía disipada en la red, PEGASIS es superior a LEACH, como muestra la figura 2.5.


\section{Dificultades de esta etapa}
