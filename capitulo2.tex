% capitulo de implementación
\chapter{Diseño e Implementación}
\section{Arquitectura de la solución}
El siguiente capitulo consta de las bases necesarias para realizar la solución. Esta se compone de 2 elementos principales. 
\begin{itemize}
\item Servidor: Encargado de recodificar (si es necesario) la fuente de audio o video y segmentarla para su distribución.
\item Cliente: Aplicación ejecutada en iOS, encargada de comunicarse al servidor para pedir el flujo de video y entregar información en twitter.
\end{itemize}
% poner figura aqui
La figura presenta una idea general del sistema.\\

Respecto a los componentes principales, estos se componen de distintos modulos que serán explicados con más detalle a lo largo del capítulo.

El protocolo utilizado para entregar el contenido audiovisual es HTTP Live Streaming, los motivos de esta elección se deben a requerimientos obligatorios designados por Apple para su sistema operativo iOS.

%en resumen un acercamiento a como funciona la cosa

%\part{Primera parte}
\section{Protocolo HTTP Live Stream}
	\subsection{Especificación}
		\subsubsection{Draft Apple}
		\subsubsection{Caracteristicas a utilizar}
			
	\subsection{Herramientas dispuestas por Apple}
		\subsubsection{id3Tag Generator}
		\subsubsection{MediaStream Segmenter}
		\subsubsection{MediaFile Segmenter}
	\subsection{Ejemplo de transmisión}
\clearpage	
\section{Reproducción y Control de Stream de Video}
	\subsection{Servidor HTTP}
		\subsubsection{Obtención del Stream}	
		\subsubsection{Sementación del Video}
		\subsubsection{Playlist Dispatcher}
	\subsection{Cliente iOS}
		\subsubsection{Recopilación de datos del Stream}
		\subsubsection{Reproducción con AV Framework}
	\subsection{Intercambio de información entre componentes}
\clearpage
\section{Interfaz Gráfica}
	\subsection{Apple Design Guidelines}
	\subsection{Componentes de UIKit}
		\subsubsection{UIDatePicker}
		\subsubsection{UIButton}
		\subsubsection{...etc}
\clearpage
\section{Integración Redes Sociales}
	\subsection{Twitter}
		\subsubsection{API Twitter}
		\subsubsection{Sharekit vs. Twitter Framework}
		\subsubsection{Post en Twitter}
		\subsubsection{Seguimiento de Hashtag}
		\subsubsection{Extracción de información de un Tweet}
	\subsection{Bit.ly}
		\subsubsection{API Bit.ly}
		\subsubsection{Seguimiento de un Hipervínculo}
		\subsubsection{Expansión de atajo bit.ly}
\clearpage
\section{Registro en iOS}
	\subsection{Schemes}
	\subsection{Redireccionamiento via Web}
		\subsubsection{PHP Script}
		\subsubsection{Respuestas según User Agent}