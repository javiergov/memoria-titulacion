
\chapter{Pruebas de funcionamiento}

Para revisar el buen funcionamiento de la aplicación desarrollada, se utilizaró el software dispuesto por Apple Inc,  el entorno de desarrollo Xcode y herramientas asociadas. Para obtener estas aplicaciones, es necesario poseer una cuenta de desarrollador para \textbf{iOS} o \textbf{Mac OS X}, las cuales se pueden obtener en el sitio web \url{https://developer.apple.com/programs/} luego de pagar la matrícula impuesta.\\

La cuenta de desarrollador permite acceso al repositorio de herramientas \cite{apple-repositorio} y también acceso a versiones de los sistemas operativos antes de lanzamiento.
Además permite poner a la venta las aplicaciones desarrolladas en la \textit{\textbf{App Store}}\cite{apple-appstore}.

% datos en el camino, seguimiento de un tweet, pantallazos de debug
\section{Cambio de tiempo y/o canal}

Las primeras prubeas relacionadas con el cambio de tiempo a través de peticiones al dispatcher con argumentos en la cadena de consulta se realizaron con la aplicación de OS X: \textbf{Quicktime X}, la cual permite reproducir streams que cumplen con las espeficicaciones de HTTP Live Streaming. Otra alternativa que además se encuentra disponible en otros sistemas operativos es \textbf{VideoLan VLC}. Las pruebas consistieron en modificar el argumento t de URLs del tipo:
\url{http://tsh.altavoz.net/videots/playlist.m3u8?s=grpz13&t=}



%escribir en consola el URL, probarlo en quicktime, vlc, navegador, app, wireshark para ver cookies

\section{Proceso corriendo en fondo} % background

escribir en consola los seekable time ranges, poner pausa background y al retomar reproducir, se notó que avanzaba la lista de iguañl forma. se arregó guardando la fecha al momento de poner pausa. que se llamaba al irse a background.

\section{Cambio de ancho de banda}

se especifica requerimiento a la empresa, se modifica el dispatcher con lista de variantes, se revisa en wireshark.
luego se utiliza la herramienta network link conditioner para modificar el BW en plena transmisión, revisar wireshark cuando cambia y ver en la misma app.
Luego se prueba con iPhone cambiando entre wifi y 3g

  \subsection{WiFi}
  se agrega campo en el URL para priorizar distintas variantes, con wifi video, 3g audio.
  \subsection{3G}
  con 3g 
\section{Cambio de transmisión mediante Twitter}
- se revisa usando safari desktop con script que incluye player


  \subsection{Dentro de la aplicación}
	- debug viendo el cambio de tranmisión, URL generado y definiendo el mismo canal. manejo en caso de tweet sin enlace
  \subsection{Scheme registrado en iOS}
	- clientes de twitter en iPhone, twitter y Tweetbot, debug en el método e imprimir en consola la app que lo lanzó.  
	- debug con app en background
	- debug con app cerrada y a la espera que parta por otra aplicación
  
\section{Enlaces Twitter en otros dispositivos}
	pantallazos de páginas de error, el por qué y cómo se entregan según el script
  \subsection{PC Escritorio}
    \subsubsection{Chrome}
    foto
    \subsubsection{Safari}
    foto del player
    \subsubsection{Firefox}
    foto
    \subsubsection{Opera}
    foto
    \subsubsection{Internet Explorer}
    foto
  \subsection{Android y otros móviles incompatibles}
  pantallazo del cell de la esperanza
\section{Comparación con SocialStream Flash}
mostrar player original de eduardo, la diferencia de la linea de tiempo, el salto de tiempo, la exactitud y que no es tan exacto porque los segmentos son de 10 segundos
