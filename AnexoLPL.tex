\chapter{ Low Power Listening}
El componente Low Power Listening (LPL) es una herramienta que TinyOS provee para el desarrollo de aplicaciones de bajo consumo. La principal característica de LPL es que permite reducir el consumo energético manteniendo la comunicación asincrónica. LPL reduce el consumo de energía de un mote atacando una de las acciones que desperdicia más energía en un sistema de comunicaciones, el ''Idle Listening''. En el Anexo A, sección A.3.3 se describen las modificaciones que se deben realizar a una aplicación para incorporar el componente LPL.\\

Para reducir el consumo, el componente LPL aplica un ciclo de trabajo al hardware de comunicación: la radio se deshabilita por un tiempo denominado ''Intervalo LPL''; una vez que ha transcurrido el intervalo, se enciende sólo el tiempo necesario para detectar si es que hay algún otro mote transmitiendo, es decir, si hay un carrier en el canal; si hay, la radio se enciende y el mote recibe el mensaje; si no hay, se deshabilita la radio nuevamente. Para que la comunicación asincrónica se realice correctamente, el mote transmisor  extiende el preámbulo del mensaje por al menos la duración del intervalo LPL, para así garantizar que el mote receptor a revisado el canal al menos una vez desde que se comenzó hasta que se terminó la operación de envío.

\begin{figure}[H]
 \centering
 \includegraphics[scale=0.35]{imgs/LplNoLpl.eps}
 \caption{Corriente medida en un TMoteSky con y sin utilización del componente LPL}
\end{figure}

En la figura C.1 se muestra una comparación entre el comportamiento de la corriente utilizada por el TmoteSky cuando no utiliza LPL (en rojo) y cuando utiliza (en azul). En esta medición, se utilizó un intervalo LPL de 50[ms] y en la figura se puede observar como cambia el nivel de corriente utilizado por el mote al operar con el componente LPL. Cuando el mote suspende la radio, se observa que el nivel de corriente utilizada baja a menos de 1[mA] en promedio, y cuando habilita la radio nuevamente para revisar el canal el nivel de corriente sube hasta 17[mA] aproximadamente.\\

De acuerdo con el principio de funcionamiento del LPL, el consumo energético debería variar principalmente por: la duración del intervalo LPL; la cantidad de mensajes recibidos por un mote; y la cantidad de mensajes enviados. Para analizar esto, se realizan dos experimentos: el primer experimento consiste en medir la corriente media utilizada por un mote, en función del intervalo LPL configurado en la aplicación, sin ningún mote transmitiendo; el segundo consiste en medir la corriente media utilizada en dos motes, uno que transmite mensajes y otro que recibe mensajes, con distintos intervalos LPL. Del primer experimento se obtienen los resultados que se observan en la figura C.2.

\begin{figure}[H]
 \centering
 \includegraphics[scale=0.5]{imgs/CaracLplConsumoVsIntervalo.eps}
 \caption{Gráfico de la corriente media utilizada por el TmoteSky en función el intervalo LPL}
\end{figure}

Como se observa en la figura C.2, se realizan mediciones de la corriente media para intervalos LPL que van desde los 50[ms] hasta 1000[ms], incrementándolos en 50[ms] para cada medición. En la figura se observa que a medida que se incrementa el intervalo LPL, la corriente media disminuye: para un intervalo LPL de 50 [ms], resulta una corriente media de 3.15[mA]; para un intervalo LPL de 500[ms], la corriente media resultante fue 0.7558[mA]; para un intervalo LPL de 1000[ms], la corriente media resultante fue 0.67[mA]. La disminución no se produce de manera proporcional, sino que más bien, parece disminuir exponencialmente. Según los resultados obtenidos, utilizando un intervalo LPL de 50[ms] y dos baterías alcalinas AA, un TMoteSky podría funcionar por 31 días. Con la misma batería y un intervalo LPL de 500[ms], el dispositivo podría funcionar por 132 días. Utilizando dos baterías alcalinas D con un intervalo LPL de 500[ms], el dispositivo podría funcionar por 661 días. Con la misma batería y utilizando un intervalo LPL de 1000[ms], el dispositivo podría funcionar por 741 días. Sin embargo, esto experimento no contempla envío ni recepción de mensajes por parte del dispositivo, operaciones que aumentan considerablemente el consumos de energía.\\

%Segundo Experimento
Para medir el impacto energético de las operaciones de envío y recepción de mensajes, se implementan dos aplicaciones de prueba, una que recibe mensajes y otra que envía mensajes periódicamente, ambas utilizando el componente LPL. El segundo experimento consiste en medir la corriente consumida por un TMoteSky que envía mensajes periódicamente utilizando LPL (con el dispositivo receptor funcionando) y, posteriormente, medir el consumo en el dispositivo receptor.

\begin{figure}[H]
 \centering
\includegraphics[scale = 0.5]{imgs/CaractLplCorrienteTx.eps}
\caption{Corriente media utilizada por el transmisor de un mensaje para distintos intervalos de envío de mensajes y distintos intervalos LPL }
\end{figure}

La figura C.3 muestra un gráfico que corresponde a la corriente media utilizada por el dispositivo transmisor para tres intervalos LPL. Para cada intervalo LPL, se utilizaron además tres intervalos para la generación de mensajes: 100, 500 y 1000 milisegundos. Se observa que para todos los intervalos LPL, la corriente media medida disminuye al aumentar el intervalo de generación de paquetes. Además, se observa que la corriente media utilizada aumenta al aumentar la duración del intervalo LPL. Esto se debe al principio del funcionamiento del componente LPL, ya que este obliga al transmisor a mantener la radio encendida un tiempo igual o mayor al de duración del intervalo LPL del receptor, con el uso intensivo de corriente que esto conlleva.

\begin{figure}[H]
 \centering
\includegraphics[scale = 0.5]{imgs/CaractLplCorrienteRx.eps}
\caption{Corriente media consumida por el receptor de un mensaje para distintos intervalos de generación de mensajes y distintos intervalos LPL}
\end{figure}

En la figura C.4, se muestra un gráfico que corresponde a la corriente media utilizada por el receptor. Se observa que la corriente media, en función  del intervalo de generación de mensajes, tiene el mismo comportamiento que en el transmisor, es decir, disminuye a medida que se aumenta la duración del intervalo. Sin embargo, se observa que la corriente media en función el intervalo LPL tiene el comportamiento inverso al que se observa en el transmisor, es decir, la corriente media disminuye a medida que se incrementa la duración del intervalo LPL.

Con estos experimentos, se observa claramente que existe un compromiso en la elección de un intervalo LPL: un intervalo muy corto aumenta la corriente media utilizada para la operación de escuchar mensajes, pero mantiene las operaciones de transmitir con una corriente media baja; un intervalo muy largo, disminuye la corriente media utilizada para escuchar mensajes, pero incrementa la corriente media utilizada para las operaciones de transmisión. Por otro lado, se observa que a medida que se aumenta la duración del intervalo de generación de mensajes, la corriente media utilizada siempre disminuye, por lo que para disminuir el consumo de energía de un mote se debe utilizar el intervalo de generación de mensajes más largo que permita la aplicación. Finalmente, se observa que LPL puede ayudar a reducir la corriente utilizada por los motes a menos de 3 [mA], que permite una autonomía de los motes del orden de meses (utilizando como referencia una batería AA), y manteniendo la comunicación asincrónica. Observando los resultados obtenidos, se plantea que para el futuro se puede desarrollar un modelo matemático para estimar el consumo de un mote, con sólo conocer ciertos parámetros del funcionamiento de la aplicación y del componente LPL; además, se debe investigar como es afectado el nivel de corriente utilizado por un mote frente a tráfico ajeno en el canal (tráfico no direccionado, ni enviado por el mote).